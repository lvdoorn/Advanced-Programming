\documentclass{article}

\title{Advanced Programming Assignment 1 Report}
\author{Mihai Popovici \texttt{MTF422} \and Leendert van Doorn \texttt{XBT504}}

\begin{document}

\maketitle	

\section{Main Design}
We have chosen to use different processes for each server, room and active room. When \texttt{kaboose:start()} is called, we call the function \texttt{loop/0}, which waits for the message \texttt{\{From, new\_room\}}, and will return the \texttt{Pid} of a new room when it receives this message.

Rooms are implemented in the \texttt{roomLoop/1} function. This function takes a list of questions as its parameter. This list is initially empty. The \texttt{roomLoop/1} function waits to receive one of three types of messages: \texttt{\{From, \{add\_question, Question\}\}}, \texttt{\{From, get\_questions\}} or \texttt{\{From, play\}}. When one of these messages is received, \texttt{roomLoop/1} will add a question to the list of questions, return the list of questions, or spawn an active room process respectively.

Active rooms are also implemented as separate processes. The state for each active room is kept in the parameters of the \texttt{activeRoomLoop/9} function. These parameters are: \texttt{Questions} = the list of questions, \texttt{Players} = map from \texttt{Ref} to \texttt{\{Nickname, true/false (active or not)\}}, \texttt{CRef} = conductor's ref, \texttt{Active} = boolean indicating if the current question is active, \texttt{Dist} = list of counters, one for each answer, \texttt{LastQ} = map from \texttt{Ref} to counter, \texttt{Total} = map from \texttt{Ref} to counter, \texttt{Time} = time in nanoseconds since question was last made active, and \texttt{HaveGuessed} = a list of all refs that have guessed.

To keep the score, we have copied and adapted the \texttt{counter} module shown during the lectures. For every answer we spawn a counter process, as well as for the scores of the current question and the total scores.

When messages are sent to an active room's process, the message is processed using the state parameters, and the appropriate parameters are given to the recursive \texttt{activeRoomLoop/9} call. The exact design choices are described below.

\section{Design Choices}
Score of inactive players is still displayed.
No validation if rejoining/leaving players actually exist
	
	
\section{Assessment of Implementation}
\end{document}