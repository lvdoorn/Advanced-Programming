\documentclass{article}

\title{Advanced Programming Assignment 1 Report}
\author{Mihai Popovici \texttt{MTF422} \and Leendert van Doorn \texttt{XBT504}}

\begin{document}

\maketitle	

\section{How to Run the Code}
To run the code, first we need to compile it. In the \texttt{src} folder we start erlang shell by running \texttt{erl} command, and then we compile following the files:
\begin{itemize}
	\item \texttt{c(kaboose).}
	\item \texttt{c(counter).}
	\item \texttt{c(kaboose\_tests).}
\end{itemize} 
Now we are ready to run the code. Once in erlang shell, we run code by specifying \texttt{module:function} and pass parameters if any, functions being situated under \texttt{kaboose.erl} (kaboose module), \texttt{kaboose\_tests.erl} (kaboose\_tests module), \texttt{counter.erl} (counter module) files.

To run tests, in the erlang shell we simply run \texttt{eunit:test(kaboose).} command.

\section{Main Design}
We have chosen to use different processes for each server, room and active room. When \texttt{kaboose:start()} is called, we call the function \texttt{loop/0}, which waits for the message \texttt{\{From, new\_room\}}, and will return the \texttt{Pid} of a new room when it receives this message.

Rooms are implemented in the \texttt{roomLoop/1} function. This function takes a list of questions as its parameter. This list is initially empty. The \texttt{roomLoop/1} function waits to receive one of three types of messages: \texttt{\{From, \{add\_question, Question\}\}}, \texttt{\{From, get\_questions\}} or \texttt{\{From, play\}}. When one of these messages is received, \texttt{roomLoop/1} will add a question to the list of questions, return the list of questions, or spawn an active room process respectively.

Active rooms are also implemented as separate processes. The state for each active room is kept in the parameters of the \texttt{activeRoomLoop/9} function. These parameters are: \texttt{Questions} = the list of questions, \texttt{Players} = map from \texttt{Ref} to \texttt{\{Nickname, true/false (active or not)\}}, \texttt{CRef} = conductor's ref, \texttt{Active} = boolean indicating if the current question is active, \texttt{Dist} = list of counters, one for each answer, \texttt{LastQ} = map from \texttt{Ref} to counter, \texttt{Total} = map from \texttt{Ref} to counter, \texttt{Time} = time in nanoseconds since question was last made active, and \texttt{HaveGuessed} = a list of all refs that have guessed.

To keep the score, we have copied and adapted the \texttt{counter} module shown during the lectures. For every answer we spawn a counter process, as well as for the scores of the current question and the total scores.

When messages are sent to an active room's process, the message is processed using the state parameters, and the appropriate parameters are given to the recursive \texttt{activeRoomLoop/9} call. The exact design choices are described below.

\section{Design Choices}
In this section we describe the edge cases that can occur when playing Kaboose!, and we describe the ways in which our implementation handles them.

\subsection{Guess Sent Without An Active Question}
Whenever a \texttt{\{From, \{guess, Ref, Index\}\}} message is sent to \texttt{activeRoomLoop/9}, the time between the moment the question was made active and the time when the guess was made is calculated to compute the score. Then, the validity of the guess is checked in \texttt{isGuessValid/5}, which uses information on the answer list, the index, if the question is active, who guessed and the list of those who have already guessed to decide whether to process or whether to ignore the guess. In case the guess is sent while the current question is inactive, the guess is ignored because the \texttt{Active} state parameter is \texttt{false}.

\subsection{Scores of Players Who Have Left}
It may happen that players of the game leave while the game has not yet finished. If this happens, we do keep displaying their scores, because they may join again at some later time.

\subsection{Nonexistent Players Leaving or Rejoining}
We do not validate whether \texttt{Ref}s that send \texttt{leave} or \texttt{rejoin} messages actually exist. This means that our program will crash if players who have never joined a game leave or rejoin it.

\subsection{Rooms Without Questions}
It is not possible in our implementation to \texttt{play/1} in a room without questions. In this case, are program will return a  \texttt{function\_clause} error, since the questions parameter of \texttt{activeRoomLoop/9} is given as \texttt{Questions = [{Description, Answers}|T]}, meaning that an empty list will not be matched.

\subsection{Unexpected Errors}
Our implementation only returns errors that were specified in the assignment text. When processes terminate in an unexpected manner, in general we do not respawn them, but the program will crash.

\section{Assessment of Implementation}
Our implementation covers all functionality describe in the assignment text. We say this with confidence based on our test suite of 35 tests, which tests all points of the specification.

Of course, ``testing can only show the presence of bugs", so we cannot be completely sure of the correct workings of our code. However, because we have decided not to cover every edge case, but only those specified in the assignment text, the number of untested cases is limited.

Furthermore, \texttt{dialyzer} does not give any warnings for our code.

We do not base our assessment on OnlineTA, because it only runs 3 tests for this assignment.

\end{document}