\documentclass{article}

\title{Advanced Programming Assignment 5 Report}
\author{Mihai Popovici \texttt{MTF422} \and Leendert van Doorn \texttt{XBT504}}

\begin{document}
	
	\maketitle	
	
	\section{How to Run the Code}
	To run the code, first we need to compile it. In the \texttt{src} folder we start the Erlang shell by running \texttt{erl} command, and then we compile everything with \texttt{cover:compile\_directory().}
	Now we are ready to run the code by specifying \texttt{module:function} and pass parameters if any.
	
	To run tests, in the Erlang shell we simply run the \texttt{eunit:test(flamingo).} command.
	
	\section{Main Design}
	We have used the \texttt{gen\_server} module to build our implementation. 
	
	When a new route is registered, we first check the prefixes to see if they are valid. If they are not, we send back an error. If they are valid, we spawn a supervisor loop that supervises a \texttt{routing\_group} process. 
	
	
	
	
	
	
	
	
	
	\section{Design Choices}

	
	\section{Assessment of Implementation}
	
	
\end{document}